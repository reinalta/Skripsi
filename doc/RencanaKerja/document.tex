\documentclass[a4paper,twoside]{article}
\usepackage[T1]{fontenc}
\usepackage[bahasa]{babel}
\usepackage{graphicx}
\usepackage{graphics}
\usepackage{float}
\usepackage[cm]{fullpage}
\pagestyle{myheadings}
\usepackage{etoolbox}
\usepackage{setspace} 
\usepackage{lipsum} 
\setlength{\headsep}{30pt}
\usepackage[inner=2cm,outer=2.5cm,top=2.5cm,bottom=2cm]{geometry} %margin
% \pagestyle{empty}

\makeatletter
\renewcommand{\@maketitle} {\begin{center} {\LARGE \textbf{ \textsc{\@title}} \par} \bigskip {\large \textbf{\textsc{\@author}} }\end{center} }
\renewcommand{\thispagestyle}[1]{}
\markright{\textbf{\textsc{AIF401/AIF402 \textemdash Rencana Kerja Skripsi \textemdash Sem. Ganjil 2021/2022}}}

\newcommand{\HRule}{\rule{\linewidth}{0.4mm}}
\renewcommand{\baselinestretch}{1}
\setlength{\parindent}{0 pt}
\setlength{\parskip}{6 pt}

\onehalfspacing
 
\begin{document}

\title{\@judultopik}
\author{\nama \textendash \@npm} 

%tulis nama dan NPM anda di sini:
\newcommand{\nama}{Reinalta Sugianto}
\newcommand{\@npm}{2017730035}
\newcommand{\@judultopik}{Perekaman Kehadiran Daring Otomatis} % Judul/topik anda
\newcommand{\jumpemb}{1} % Jumlah pembimbing, 1 atau 2
\newcommand{\tanggal}{04/10/2021}

% Dokumen hasil template ini harus dicetak bolak-balik !!!!

\maketitle

\pagenumbering{arabic}

\section{Deskripsi}
Sebelum adanya pandemi, perekaman kehadiran perkuliahan dilakukan secara fisik. Ada beberapa cara perekaman kehadiran perkulihan, seperti menggunakan \textit{fingerprint} atau dicatat langsung oleh dosen bagi mahasiswa. Pada masa pandemi ini perekaman kehadiran di UNPAR dilakukan dengan menggunakan aplikasi atau situs web. Cara perekaman kehadiran di UNPAR ini mumbutuhkan waktu lebih agar dapat tercatat perekaman kehadirannya, karena butuh waktu untuk membuka situs web serta perlu memasukan \textit{email} dan \textit{password}. Perekaman kehadiran daring otomatis ini akan menggunakan Selenium WebDriver.

Selenium WebDriver adalah sebuah \textit{tools} yang berguna untuk melakukan otomatisasi terhadap web pada browser. Selenium WebDriver ini tersedia untuk bahasa permrograman Ruby, Java, Python, C\#, dan JavaScript. Skripsi ini dibuat untuk mahasiswa dapat melakukan perekaman kehadiran dengan satu "klik". Pengertian dari satu "klik" ini adalah untuk mengurangi waktu yang dibutuhkan mahasiswa berinteraksi dengan aplikasi atau situs web UNPAR, bukan untuk mempercepat waktu agar kehadiran terekam. 

\section{Rumusan Masalah}
Rumusan masalah yang akan dibahas di skripsi ini adalah sebagai berikut :
\begin{itemize}
	\item Fitur apa saja yang akan tersedia di program Perekaman Kehadiran Daring Otomatis?
	\item Bagaimana membangun program Perekaman Kehadiran Daring Otomatis?
	\item Bagaimana membuat sebuah program yang mampu menerima rangsangan satu tombol Perekaman Kehadiran Daring Otomatis?
	
\end{itemize}

\section{Tujuan}
Tujuan yang ingin dicapai dari penulisan skripsi ini sebagai berikut :
\begin{itemize}
	\item Mendefinisikan fitur apa saja yang tersedia di program Perekaman Kehadiran Daring Otomatis.
	\item Membangun program menggunakan Selenium WebDriver.
	\item Membuat program yang dapat membuka web sampai melakukan rekam kehadiran secara otomatis.
	
\end{itemize}



\section{Deskripsi Perangkat Lunak}
Perangkat lunak akhir yang akan dibuat memiliki fitur minimal sebagai berikut:
\begin{itemize}
	\item Membuka web secara otomatis.
	\item Membuka Student Portal UNPAR secara otomatis.
	\item Memasukan \textit{username} dan \textit{password}.
	\item Melakukan rekam kehadiran.
	\item Fitur lain yang nantinya dibutuhkan setelah analisis akan ditambahkan. 
\end{itemize}

\section{Detail Pengerjaan Skripsi}
Bagian-bagian pekerjaan skripsi ini adalah sebagai berikut :
	\begin{enumerate}
		\item Melakukan studi mengenai Selenium WebDriver.
		\item Menganalisis web Student Portal UNPAR.
		\item Mempelajari bahasa pemrograman yang akan digunakan.
		\item Membangun program perekaman kehadiran daring otomatis.
		\item Melakukan pengujian dan eksperimen.
		\item Menulis dokumen skripsi.		

	\end{enumerate}

\section{Rencana Kerja}
Rincian capaian yang direncanakan di Skripsi 1 adalah sebagai berikut:
\begin{enumerate}
\item Mempelajari mengenai Selenium WebDriver.
\item Menganalisis web Studend Portal UNPAR.
\item Menulis dokumen skripsi (Bab 1-3).
\end{enumerate}

Sedangkan yang akan diselesaikan di Skripsi 2 adalah sebagai berikut:
\begin{enumerate}
\item Membuat program perekaman kehadiran daring otomatis.
\item Melakukan pengujian dan eksperimen terhadap program yang dibuat.
\item Menulis dokumen skripsi (Bab 4-6).
\end{enumerate}

\vspace{0.05cm} %\semula 1cm
\centering Bandung, \tanggal\\
\vspace{2cm} \nama \\ 
\vspace{1cm}

Menyetujui, \\
\ifdefstring{\jumpemb}{2}{
\vspace{1.5cm}
\begin{centering} Menyetujui,\\ \end{centering} \vspace{0.75cm}
\begin{minipage}[b]{0.45\linewidth}
% \centering Bandung, \makebox[0.5cm]{\hrulefill}/\makebox[0.5cm]{\hrulefill}/2013 \\
\vspace{2cm} Nama: \makebox[3cm]{\hrulefill}\\ Pembimbing Utama
\end{minipage} \hspace{0.5cm}
\begin{minipage}[b]{0.45\linewidth}
% \centering Bandung, \makebox[0.5cm]{\hrulefill}/\makebox[0.5cm]{\hrulefill}/2013\\
\vspace{2cm} Nama: \makebox[3cm]{\hrulefill}\\ Pembimbing Pendamping
\end{minipage}
\vspace{0.5cm}
}{
% \centering Bandung, \makebox[0.5cm]{\hrulefill}/\makebox[0.5cm]{\hrulefill}/2013\\
\vspace{2cm} Nama: \makebox[3cm]{\hrulefill}\\ Pembimbing Tunggal
}
\end{document}

