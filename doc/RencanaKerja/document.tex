\documentclass[a4paper,twoside]{article}
\usepackage[T1]{fontenc}
\usepackage[bahasa]{babel}
\usepackage{graphicx}
\usepackage{graphics}
\usepackage{float}
\usepackage[cm]{fullpage}
\pagestyle{myheadings}
\usepackage{etoolbox}
\usepackage{setspace} 
\usepackage{lipsum} 
\setlength{\headsep}{30pt}
\usepackage[inner=2cm,outer=2.5cm,top=2.5cm,bottom=2cm]{geometry} %margin
% \pagestyle{empty}

\makeatletter
\renewcommand{\@maketitle} {\begin{center} {\LARGE \textbf{ \textsc{\@title}} \par} \bigskip {\large \textbf{\textsc{\@author}} }\end{center} }
\renewcommand{\thispagestyle}[1]{}
\markright{\textbf{\textsc{AIF401/AIF402 \textemdash Rencana Kerja Skripsi \textemdash Sem. Ganjil 2021/2022}}}

\newcommand{\HRule}{\rule{\linewidth}{0.4mm}}
\renewcommand{\baselinestretch}{1}
\setlength{\parindent}{0 pt}
\setlength{\parskip}{6 pt}

\onehalfspacing
 
\begin{document}

\title{\@judultopik}
\author{\nama \textendash \@npm} 

%tulis nama dan NPM anda di sini:
\newcommand{\nama}{Reinalta Sugianto}
\newcommand{\@npm}{2017730035}
\newcommand{\@judultopik}{Perekaman Kehadiran Daring Otomatis} % Judul/topik anda
\newcommand{\jumpemb}{1} % Jumlah pembimbing, 1 atau 2
\newcommand{\tanggal}{04/10/2021}

% Dokumen hasil template ini harus dicetak bolak-balik !!!!

\maketitle

\pagenumbering{arabic}

\section{Deskripsi}
Sebelum adanya pandemi, perekaman kehadiran perkuliahan dilakukan secara fisik. Ada beberapa cara perekaman kehadiran perkulihan, seperti menggunakan \textit{fingerprint} atau dicatat langsung oleh dosen bagi mahasiswa. Pada masa pandemi ini perekaman kehadiran di UNPAR dilakukan dengan menggunakan aplikasi atau situs web. Cara perekaman kehadiran di UNPAR ini mumbutuhkan waktu lebih agar dapat tercatat perekaman kehadirannya, karena butuh waktu untuk membuka situs web serta perlu memasukan \textit{email} dan \textit{password}. Perekaman kehadiran daring otomatis ini akan menggunakan Selenium WebDriver.

Selenium WebDriver adalah sebuah \textit{tools} yang berguna untuk melakukan otomatisasi terhadap web pada browser. Selenium WebDriver ini tersedia untuk bahasa permrograman Ruby, Java, Python, C#, dan JavaScript. Skripsi ini dibuat untuk mahasiswa dapat melakukan perekaman kehadiran dengan satu "klik" . Pengertian dari satu "klik" ini adalah untuk mengurangi waktu yang dibutuhkan mahasiswa berinteraksi dengan aplikasi atau situs web UNPAR, bukan untuk mempercepat waktu agar kehadiran terekam. 

\section{Rumusan Masalah}
Tuliskan rumusan dari masalah yang akan anda bahas pada skripsi ini. Rumusan masalah biasanya berupa kalimat pertanyaan. Gunakan itemize seperti contoh di bagian Deskripsi Perangkat Lunak.

\section{Tujuan}
Tuliskan tujuan dari topik skripsi yang anda ajukan. Tujuan penelitian biasanya berkaitan erat dengan pertanyaan yang diajukan di bagian rumusan masalah. Gunakan itemize seperti contoh di bagian Deskripsi Perangkat Lunak.

\section{Deskripsi Perangkat Lunak}
Tuliskan deksripsi dari perangkat lunak yang akan anda hasilkan. Apa saja fitur yang disediakan oleh PL tersebut dan apa saja kemampuan dari PL tersebut. Perhatikan contoh di bawah ini:

Perangkat lunak akhir yang akan dibuat memiliki fitur minimal sebagai berikut:
\begin{itemize}
	\item Pengguna dapat melihat denah Musem Geologi Bandung dalam bidang dua dimensi. Sedangkan pengunjung direpresentasikan menggunakan lingkaran-lingkaran kecil (tidak menggunakan gambar manusia yang diambil dari atas)
	\item Pengguna dapat memunculkan atau menghilangkan gambar {\it flow tiles} pada denah museum. 
	\item Pengguna dapat mengatur jalannya simulasi: memulai(start) simulasi, menunda(pause) simulasi, melanjutkan(continue) simulasi, maupun menghentikan(stop) simulasi
	\item Pengguna dapat mengatur banyaknya pengunjung di dalam museum, baik melalui perubahan frekuensi kedatangan pengunjung maupun menambahkan dan menghapus pengunjung satu-persatu secara manual.
	\item Posisi kamera dapat diubah (pergerakan di bidang tiga dimensi) sehingga pengguna dapat melihat simulasi di museum dari berbagai arah. 
	\item Posisi kamera dapat diubah untuk emngikuti perjalanan seorang pengunjung di dalam 
	\item Pengguna dapat memilih apakah akan menggunakan teknik {\it flow tiles} atau tidak pada saat simulasi berlangsung
	\item Jenis {\it flow tiles} yang digunakan dapat diubah-ubah pada saat simulasi sedang berlangsung
		
\end{itemize}

\section{Detail Pengerjaan Skripsi}
Tuliskan bagian-bagian pengerjaan skripsi secara detail. Bagian pekerjaan tersebut mencakup awal hingga akhir skripsi, termasuk di dalamnya pengerjaan dokumentasi skripsi, pengujian, survei, dll.

Bagian-bagian pekerjaan skripsi ini adalah sebagai berikut :
	\begin{enumerate}
		\item Melakukan survei ke Museum Geologi Bandung untuk mendapatkan denah serta mengetahui perilaku pengunjung museum secara umum (arah perjalanan, kecepatan, lama melihat objek, dll)
		\item Melakukan analisis pada hasil survei terhadap pergerakan pengunjung di museum dan membuat rancangan denah di komputer yang dilengkapi dengan penghalang dan objek di museum.
		\item Melakukan studi literatur mengenai sifat kolektif suatu kerumunan, teknik {\it social force model} dan teknik {\it flow tiles}
		\item Mempelajari bahasa pemrograman C++ dan cara menggunakan framework OpenSteer
		\item Merancang pergerakan kerumunan di dalam museum menggunakan teknik {\it social force model} dan {\it flow tiles} serta menggunakan teknik lainnya seperti konsep pathway dan waypoints. Selain itu, dirancang pula adanya waktu tunggu (pada saat pengunjung melihat objek di museum) dan cara pembuatan jalur bagi setiap individu pengunjung
		\item Melakukan analisa dan merancang struktur data yang cocok untuk menyimpan penghalang (obstacle)
		\item Mengimplementasikan keseluruhan algoritma dan struktur data yang dirancang, dengan menggunakan framework OpenSteer 
		\item Melakukan pengujian (dan eksperimen) yang melibatkan responde untuk menilai hasil simulasi secara kualitatif
		\item Menulis dokumen skripsi
	\end{enumerate}

\section{Rencana Kerja}
Rincian capaian yang direncanakan di Skripsi 1 adalah sebagai berikut:
\begin{enumerate}
\item
\item
\item
\end{enumerate}

Sedangkan yang akan diselesaikan di Skripsi 2 adalah sebagai berikut:
\begin{enumerate}
\item
\item
\item
\end{enumerate}

\vspace{1cm}
\centering Bandung, \tanggal\\
\vspace{2cm} \nama \\ 
\vspace{1cm}

Menyetujui, \\
\ifdefstring{\jumpemb}{2}{
\vspace{1.5cm}
\begin{centering} Menyetujui,\\ \end{centering} \vspace{0.75cm}
\begin{minipage}[b]{0.45\linewidth}
% \centering Bandung, \makebox[0.5cm]{\hrulefill}/\makebox[0.5cm]{\hrulefill}/2013 \\
\vspace{2cm} Nama: \makebox[3cm]{\hrulefill}\\ Pembimbing Utama
\end{minipage} \hspace{0.5cm}
\begin{minipage}[b]{0.45\linewidth}
% \centering Bandung, \makebox[0.5cm]{\hrulefill}/\makebox[0.5cm]{\hrulefill}/2013\\
\vspace{2cm} Nama: \makebox[3cm]{\hrulefill}\\ Pembimbing Pendamping
\end{minipage}
\vspace{0.5cm}
}{
% \centering Bandung, \makebox[0.5cm]{\hrulefill}/\makebox[0.5cm]{\hrulefill}/2013\\
\vspace{2cm} Nama: \makebox[3cm]{\hrulefill}\\ Pembimbing Tunggal
}
\end{document}

