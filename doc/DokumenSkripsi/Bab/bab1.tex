%versi 2 (8-10-2016) 
\chapter{Pendahuluan}
\label{chap:intro}
   
\section{Latar Belakang}
\label{sec:label}
Perkuliahan di UNPAR biasanya membutuhkan perekaman kehadiran untuk mengetahui kehadiran mahasiswa dan dosen, bagi mahasiswa UNPAR perekaman kehadiran biasanya dilakukan dengan melakukan tanda tangan pada daftar kehadiran atau dicatat langsung oleh dosen yang memanggil mahasiswanya, sedangkan bagi dosen UNPAR perekaman kehadiran dilakukan dengan menggunakan  \textit{fingerprint}. Perekaman kehadiran diperkirakan membutuhkan waktu sekitar kurang dari 5 detik.

Pada tahun 2020 terjadi pandemi Covid-19 di seluruh negara. Pandemi Covid-19 masuk ke Indonesia pada awal bulan Maret tahun 2020. Covid-19 adalah penyakit yang disebabkan oleh virus \textit{severe acute respiratory syndrome coronavirus 2} (SARS-CoV-2) \footnote{\textit{Pandemi Covid-19 di Indonesia} \url{https://id.wikipedia.org/wiki/Pandemi_Covid-19_di_Indonesia}}. Penularan virus Covid-19 terjadi saat seseorang menyentuh barang yang sudah terkontaminasi oleh droplet orang yang terkena virus Covid-19 atau terkena droplet orang lain saat berinteraksi langsung dengan orang yang terkena virus Covid-19.  Akibat pandemi Covid-19 yang dapat menular ini, maka hampir seluruh kegiatan di Indonesia dilakukan secara daring untuk mengurangi interaksi orang secara langsung yang dapat meningkatkan angka penularan virus tersebut. 

Pembelajaran secara daring diberlakukan oleh UNPAR di akhir bulan Maret untuk seluruh kegiatan perkuliahan demi mencegah penularan virus Covid-19. Akibat diberlakukannya pembelajaran secara daring, maka perekaman kehadiran di UNPAR dilakukan dengan menggunakan aplikasi atau situs web milik UNPAR. Cara perekaman kehadiran secara daring di UNPAR ini mumbutuhkan waktu lebih agar dapat tercatat perekaman kehadirannya, karena butuh waktu untuk membuka situs web serta perlu memasukan \textit{email} dan \textit{password} hingga akhirnya melakukan perekaman kehadiran. 

Selenium adalah \textit{open-source} \textit{framework} pengujian otomatisasi untuk aplikasi web\cite{selenium}. Cara kerja Selenium untuk melakukan otomatisasi pada browser web adalah seperti menirukan interaksi pengguna dengan browser, yang nantinya akan dijalankan otomatis oleh Selenium. WebDriver adalah \textit{Application Programming Interface} (API) yang berfungsi menghubungkan Selenium dengan browser web, sehingga Selenium dapat mengontrol atau melakukan otomatisasi pada browser web. WebDriver ini seolah-olah membuat pengguna secara langsung mengoperasi \textit{browser}, padahal dijalankan secara otomatis langsung oleh WebDriver tersebut. Selenium ini tersedia untuk bahasa pemrograman Ruby, Java, Python, C\#, dan JavaScript. Pembuatan Perekaman kehadiran daring otomatis ini akan menggunakan Selenium dengan bahasa pemrograman Python.  
\newpage
Proses perekaman kehadiran daring di UNPAR harus melakukan beberapa hal untuk dapat melakukan perekeman kehadiran daring untuk mata kuliah yang diambil. Berikut ini hal-hal yang perlu dilakukan untuk melakukan perekaman kehadiran daring di UNPAR:
\begin{enumerate}
	\item Membuka browser.
	\item Membuka situs \url{https://studentportal.unpar.ac.id}.
	\item Mengisi \textit{email} dan \textit{password} mahasiswa.
	\item Menuju ke halaman web untuk perekaman kehadiran mahasiswa.
	\item Melakukan rekam kehadiran.
\end{enumerate}
Pada skripsi ini, akan dibuat sebuah program yang dapat melakukan perekaman kehadiran otomatis dengan sistem menerima rangsangan satu ``klik'' yang dapat menjalankan langkah-langkah perekaman kehadiran daring manual secara otomatis pada situs \url{https://studentportal.unpar.ac.id}, sehingga program yang dibuat ini menjalankan perintah yang biasa dilakukan mahasiswa lakukan secara manual untuk melakukan perekaman kehadiran daring menjadi otomatis dilakukan oleh program tersebut.
program ini bertujuan agar mahasiswa dapat melakukan perekaman kehadiran secara online di situs web Portal Akademik Mahasiswa UNPAR dengan lebih mudah dikarenakan mahasiswa hanya perlu menjalankan program tersebut untuk melakukan perekaman kehadiran serta mengurangi waktu yang dibutuhkan untuk berinteraksi dengan aplikasi atau situs web dan bukan untuk mempercepat waktu agar kehadiran terekam, sehingga membuat waktu perekaman kehadiran secara daring dapat mendekati atau menyamai waktu perekaman kehadiran secara luring. Dikarenakan terbimbing tidak memiliki akses ke \url{https://akuhadir.unpar.ac.id} situs perekaman kehadiran milik dosen, maka terbimbing mensimulasikan dengan Portal Akademik Mahasiswa dan kemudian Pembimbing mengubah aksesnya ke situs perekeman kehadiran milik dosen.


\section{Rumusan Masalah}
\label{sec:rumusan}
Rumusan masalah yang akan dibahas di skripsi ini adalah sebagai berikut :
\begin{enumerate}
	\item Bagaimana cara membangun program perekaman kehadiran daring otomatis?
	\item Bagaimana cara mengurangi waktu interaksi dengan aplikasi atau situs web untuk merekam kehadiran secara otomatis?
\end{enumerate}

\section{Tujuan}
\label{sec:tujuan}
Tujuan yang ingin dicapai dari penulisan skripsi ini sebagai berikut :
\begin{enumerate}
	\item Membangun program perekaman kehadiran daring otomatis dengan Selenium WebDriver.
	\item Membuat program yang mampu mengurangi waktu interaksi dengan aplikasi atau situs web untuk merekam kehadiran secara otomatis.
\end{enumerate}

\section{Batasan Masalah}
\label{sec:batasan}
Beberapa batasan yang dibuat terkait dengan pengerjaan skripsi ini adalah sebagai berikut :
\begin{enumerate}
	\item Program ini bukan untuk mempercepat kehadiran terekam, hanya untuk mengurangi waktu untuk berinteraksi dengan aplikasi.
	\item Pengguna harus melakukan \textit{install} Python3 dan mengeksekusi progamnya melalui \textit{Command Prompt}.
\end{enumerate}


\section{Metodologi}
\label{sec:metlit}
Metodologi yang dilakukan pada skripsi ini adalah sebagai berikut :
\begin{enumerate}
	\item Melakukan studi mengenai Selenium WebDriver.
	\item Mempelajari bahasa pemrograman python.
	\item Mempelajari cara menggunakan Selenium.
	\item Menganalisis web Student Portal UNPAR.
	\item Membangun program perekaman kehadiran daring otomatis.
	\item Melakukan pengujian dan eksperimen.
	\item Menulis dokumen skripsi.		
\end{enumerate}


\section{Sistematika Pembahasan}
\label{sec:sispem}
Sistematika penulisan setiap bab skripsi ini adalah sebagai berikut :
\begin{enumerate}
	\item Bab 1 Pendahuluan \\
	Bab ini berisi latar belakang, rumusan masalah, tujuan, batasan masalah, metodologi, dan sistematika pembahasan yang digunakan untuk menyusun skripsi ini.
	\item Bab 2 Dasar Teori \\
	Bab ini berisi teori-teori yang digunakan dalam pembuatan skripsi ini. Teori yang digunakan yaitu Portal Akademik Mahasiswa, AKUHADIR, \textit{Cascading Style Sheets Selector}, Selenium, WebDriver, dan \textit{Library} Python.
	\item Bab 3 Analisis Masalah \\
	Bab ini berisi analisis yang digunakan pada skripsi ini, yaitu analisis hasil survei perekaman kehadiran daring dan luring, analisis alur perekaman kehadirang daring, cara menerjemahkan perekaman kehadiran daring ke dalam selenium, dan analisis program sejenis.
	\item Bab 4 Perancangan \\
	Bab ini berisi perancangan program, meliputi masukan program dan diagram aktivitas.
	\item Bab 5 Implementasi dan Pengujian\\
	Bab ini berisi implementasi dan pengujian program, meliputi lingkungan implementasi, hasil implementasi, pengujian fungsional, dan pengujian eksperimental.
	\item Bab 6 Kesimpulan dan Saran \\
	Bab ini berisi kesimpulan dari hasil pembangunan program beserta saran untuk pengembangan selanjutnya.
\end{enumerate}
