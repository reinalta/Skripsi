%versi 2 (8-10-2016) 
\chapter{Pendahuluan}
\label{chap:intro}
   
\section{Latar Belakang}
\label{sec:label}
Perkuliahan di UNPAR biasanya membutuhkan perekaman kehadiran untuk mengetahui kehadiran mahasiswa dan dosen, bagi mahasiswa UNPAR perekaman kehadiran biasanya dilakukan dengan melakukan tanda tangan pada daftar kehadiran atau dicatat langsung oleh dosen yang memanggil mahasiswanya, sedangkan bagi dosen UNPAR perekaman kehadiran dilakukan dengan menggunakan  \textit{fingerprint}. Perekaman kehadiran diperkirakan membutuhkan waktu sekitar kurang dari 5 detik.

Pada tahun 2020 terjadi pandemi Covid-19 di seluruh negara. Pandemi Covid-19 masuk ke Indonesia pada awal bulan Maret tahun 2020. Covid-19 adalah penyakit yang disebabkan oleh virus \textit{severe acute respiratory syndrome coronavirus 2} (SARS-CoV-2) \footnote{\textit{Pandemi Covid-19 di Indonesia} \url{https://id.wikipedia.org/wiki/Pandemi_Covid-19_di_Indonesia}}. Penularan virus Covid-19 terjadi saat seseorang menyentuh barang yang sudah terkontaminasi oleh droplet orang yang terkena virus Covid-19 atau terkena droplet orang lain saat berinteraksi langsung dengan orang yang terkena virus Covid-19.  Akibat pandemi Covid-19 yang dapat menular ini, maka hampir seluruh kegiatan di Indonesia dilakukan secara daring untuk mengurangi interaksi orang secara langsung yang dapat meningkatkan angka penularan virus tersebut. 

Pembelajaran secara daring diberlakukan oleh UNPAR di akhir bulan Maret untuk seluruh kegiatan perkuliahan demi mencegah penularan virus Covid-19. Akibat diberlakukannya pembelajaran secara daring, maka perekaman kehadiran di UNPAR dilakukan dengan menggunakan aplikasi atau situs web milik UNPAR. Cara perekaman kehadiran secara daring di UNPAR ini mumbutuhkan waktu lebih agar dapat tercatat perekaman kehadirannya, karena butuh waktu untuk membuka situs web serta perlu memasukan \textit{email} dan \textit{password} hingga akhirnya melakukan perekaman kehadiran. 

Selenium adalah \textit{open-source} \textit{framework} pengujian otomatisasi untuk aplikasi web\footnote{\textit{Selenium} \url{https://www.selenium.dev/documentation/}}. WebDriver menggunakan API otomatisasi browser yang disediakan oleh vendor browsernya untuk mengontrol browser dan melakukan pengujian. API WebDriver ini seolah-olah membuat pengguna secara langsung mengoperasi browser, padahal dijalankan secara otomatis langsung oleh \textit{API} WebDriver tersebut. Selenium WebDriver adalah sebuah \textit{tools} yang berguna untuk melakukan otomatisasi terhadap web pada browser. Selenium WebDriver ini tersedia untuk bahasa pemrograman Ruby, Java, Python, C\#, dan JavaScript. Pembuatan Perekaman kehadiran daring otomatis ini akan menggunakan Selenium WebDriver dengan bahasa pemrograman Python.  

Pada skripsi ini, akan dibuat sebuah perangkat lunak yang dapat melakukan perekaman kehadiran otomatis dengan sistem menerima rangsangan satu "klik", sehingga dapat melakukan hal-hal berikut :
\begin{enumerate}
	\item Membuka peramban.
	\item Membuka situs web perekaman kehadiran.
	\item Mengisi dan \textit{login} dengan \textit{username} serta \textit{password} yang ddiambil dari file konfigurasi.
	\item Melakukan rekam kehadiran.
\end{enumerate} 
perangkat lunak ini bertujuan agar mahasiswa dan dosen dapat melakukan perekaman kehadiran secara online dengan lebih mudah serta mengurangi waktu yang dibutuhkan untuk berinteraksi dengan aplikasi atau situs web dan bukan untuk mempercepat waktu agar kehadiran terekam, sehingga membuat waktu perekaman kehadiran secara daring dapat menyamai waktu perekaman kehadiran secara luring. 


\section{Rumusan Masalah}
\label{sec:rumusan}
Rumusan masalah yang akan dibahas di skripsi ini adalah sebagai berikut :
\begin{itemize}
	\item Bagaimana cara membangun program Perekaman Kehadiran Daring Otomatis?
	\item Bagaimana cara mengurangi waktu interaksi dengan aplikasi atau situs web untuk merekam kehadiran?
\end{itemize}

\section{Tujuan}
\label{sec:tujuan}
Tujuan yang ingin dicapai dari penulisan skripsi ini sebagai berikut :
\begin{itemize}
	\item Membangun program menggunakan Selenium WebDriver.
	\item Membuat program yang mampu menerima rangsangan satu tombol untuk melakukan beberapa hal menggunakan Selenium.
\end{itemize}

\section{Batasan Masalah}
\label{sec:batasan}
Beberapa batasan yang dibuat terkait dengan pengerjaan skripsi ini adalah sebagai berikut :
\begin{enumerate}
	\item Program ini bukan untuk mempercepat kehadiran terekam, hanya untuk mengurangi waktu untuk berinteraksi dengan aplikasi.
\end{enumerate}


\section{Metodologi}
\label{sec:metlit}
Metodologi yang dilakukan pada skripsi ini adalah sebagai berikut :
\begin{enumerate}
	\item Melakukan studi mengenai Selenium WebDriver.
	\item Mempelajari bahasa pemrograman python.
	\item Mempelajari cara menggunakan Selenium.
	\item Menganalisis web Student Portal UNPAR.
	\item Membangun program perekaman kehadiran daring otomatis.
	\item Melakukan pengujian dan eksperimen.
	\item Menulis dokumen skripsi.		
\end{enumerate}


\section{Sistematika Pembahasan}
\label{sec:sispem}
Sistematika penulisan setiap bab skripsi ini adalah sebagai berikut :
\begin{enumerate}
	\item Bab 1 Pendahuluan \\
	Bab ini berisi latar belakang, rumusan masalah, tujuan, batasan masalah, metodologi, dan sistematika pembahasan yang digunakan untuk menyusun skripsi ini.
	\item Bab 2 Dasar Teori \\
	Bab ini 
	\item Bab 3 Analisis Masalah \\
	Bab ini

\end{enumerate}
