%versi 2 (8-10-2016) 
\chapter{Pendahuluan}
\label{chap:intro}
   
\section{Latar Belakang}
\label{sec:label}
TESSTES TESS dsfadsf
Bagian ini akan diisi dengan apa yang melatarbelakangi pembuatan template skripsi ini.
Termasuk juga masalah-masalah yang akan dihadapi untuk membuatnya, termasuk kurangnya kemampuan penguasaan \LaTeX{} sehingga template ini dibuat dengan mengandalkan berbagai contoh yang tersebar di dunia maya, yang digabung-gabung menjadi satu jua.
Bagian lain juga akan dilengkapi, untuk sementara diisi dengan lorem ipsum versi bahasa inggris.


\section{Rumusan Masalah}
\label{sec:rumusan}
Rumusan masalah yang akan dibahas di skripsi ini adalah sebagai berikut :
\begin{itemize}
	\item Bagaimana cara membangun program Perekaman Kehadiran Daring Otomatis?
	\item Bagaimana cara mengurangi waktu interaksi dengan aplikasi atau situs web untuk merekam kehadiran?
\end{itemize}

\section{Tujuan}
\label{sec:tujuan}
Tujuan yang ingin dicapai dari penulisan skripsi ini sebagai berikut :
\begin{itemize}
	\item Membangun program menggunakan Selenium WebDriver.
	\item Membuat program yang mampu menerima rangsangan satu tombol untuk melakukan beberapa hal menggunakan Selenium.
\end{itemize}

\section{Batasan Masalah}
\label{sec:batasan}
Beberapa batasan yang dibuat terkait dengan pengerjaan skripsi ini adalah sebagai berikut :
\begin{enumerate}
	
\end{enumerate}


\section{Metodologi}
\label{sec:metlit}
Metodologi yang dilakukan pada skripsi ini adalah sebagai berikut :
\begin{enumerate}
	\item Melakukan studi mengenai Selenium WebDriver.
	\item Mempelajari bahasa pemrograman python.
	\item Mempelajari cara menggunakan Selenium.
	\item Menganalisis web Student Portal UNPAR.
	\item Membangun program perekaman kehadiran daring otomatis.
	\item Melakukan pengujian dan eksperimen.
	\item Menulis dokumen skripsi.		
\end{enumerate}



\section{Sistematika Pembahasan}
\label{sec:sispem}
Sistematika penulisan setiap bab skripsi ini adalah sebagai berikut :
\begin{enumerate}
	
\end{enumerate}
