\chapter{Implementasi dan Pengujian}
\label{chap:implementasidanpengujian}
Bab ini berisi Implementasi Perangkat Lunak dan Pengujian Perangkat Lunak. Bagian implementasi terdiri dari penjelasan lingkungan pengembangan perangkat lunak dan hasil implementasi. Bagian pengujian terdiri dari hasil pengujian fungsional dan eksperimental terhadap perangkat lunak yang telah dibangun.

\section{Implementasi}
\label{sec:implementasi} 

\subsection{Lingkungan Implementasi}
Implementasi perangkat lunak ini dilakukan pada komputer penulis dengan spesifikasi berikut:
\begin{enumerate}
	\item \textit{Processor}: Intel Core i5 9400F
	\item \textit{Random Access Memory} (RAM): 16 GB DDR4
	\item Sistem Operasi: Windows 10
	\item Versi Python : Python 3.8.5
\end{enumerate}

\subsection{Hasil Implementasi}
Hasil implementasi berupa sebuah perangkat lunak perekaman kehadiran daring otomatis dengan bahasa pemrograman python. Sebelum menjalankan perangkat lunak untuk perekaman kehadiran daring otomatis, terdapat \textit{file} .ini yang merupakan sebuah masukkan untuk perangkat lunak. \textit{File} .ini dibahas pada Subbab \ref{sec:inputConfig}. Contoh \textit{file} .ini dapat dilihat pada \ref{kode:5:kodemasukan}.
\begin{lstlisting}[caption=Contoh \textit{file} .ini untuk Masukan Perangkat Lunak Perekaman Kehadiran Daring Otomatis, label=kode:5:kodemasukan]
	[database_config]
	1 = open https://studentportal.unpar.ac.id
	2 = click #login-button
	3 = sendkeys #username 2017730035@student.unpar.ac.id 
	4 = click #next_login
	5 = sendkeys #password 12345
	6 = click #appPass>div.login__form>button
	7 = or a[href='https://studentportal.unpar.ac.id/jadwal'] .swal-button.swal-button--confirm.swal-button--danger
	8 = click a[onclick="absenPerkuliahan(this)"]
	9 = click .swal-button.swal-button--confirm.swal-button--danger9
	10 = quit
\end{lstlisting}

Perekaman kehadiran daring otomatis dapat dilakukan dengan menjalankan perangkat lunak. Pengguna perlu membuka \textit{Command Prompt} pada komputer maupun laptop dengan \textit{directory} file automatedTesting.py berada dan menuliskan perintah ``python automatedTesting.py'' atau ``py automatedTesting.py'' pada \textit{Command Prompt}, seperti pada tampilan Gambar \ref{fig:cmd}
\begin{figure}[H]
	\centering
	\includegraphics[scale=0.5]{Gambar/cmd.jpg}
	\caption{Tampilan \textit{Command Prompt} dengan \textit{Directory File}} 
	\label{fig:cmd}
\end{figure}

Setelah pengguna menekan ``Enter'' pada \textit{Command Prompt} maka perangkat lunak akan melakukan perekaman kehadiran daring secara otomatis, bagi mahasiswa maka perangkat lunak akan melakukan perekaman kehadiran daring pada Portal Akademik Mahasiswa secara otomatis, dimana perangkat lunak akan menjalankan secara otomatis tahap-tahap perekaman kehadiran daring secara manual yang biasa dilakukan mahasiswa seperti yang dibahas pada Subbab \ref{sec:alur}, sedangkan bagi dosen maka perangkat lunak akan melakukan perekaman kehadiran daring pada AKUHADIR seperti yang dibahas pada Subbab \ref{sec:akuhadir}. Setelah berhasil melakukan perekaman kehadiran daring maka akan muncul notifikasi bahwa perekaman berhasil dilakukan, seperti pada tampilan Gambar \ref{fig:absenBerhasil}, selain itu akan muncul notifikasi berupa peringatan bahwa absensi gagal, seperti pada tampilan Gambar \ref{fig:absenGagal}. Absensi gagal terjadi karena tidak ada jadwal kuliah lagi bagi mahasiswa, atau sudah melakukan absensi sehingga tidak ada yang bisa lagi untuk melakukan perekaman kehadiran.
\begin{figure}[H]
	\centering
	\includegraphics[scale=0.7]{Gambar/infoBox.jpg}
	\caption{Tampilan Notifikasi Berhasil Absen} 
	\label{fig:absenBerhasil}
\end{figure}

\begin{figure}[H]
	\centering
	\includegraphics[scale=0.7]{Gambar/gagalAbsen.jpg}
	\caption{Tampilan Notifikasi Gagal Absen} 
	\label{fig:absenGagal}
\end{figure}

\section{Pengujian}
\label{sec:pengujian} 

\subsection{Pengujian Fungsional Mahasiswa}
Pengujian fungsional dilakukan untuk mengetahui kesesuaian reaksi perangkat lunak dengan reaksi yang diharapkan berdasarkan aksi pengguna terhadap perangkat lunak. Tabel \ref{tab:fungsi} merupakan tabel hasil pengujian perangkat lunak pada komputer penulis dengan spesifikasi berikut:
\begin{enumerate}
	\item \textit{Processor}: Intel Core i5 9400F
	\item \textit{Random Access Memory} (RAM): 16 GB DDR4
	\item Sistem Operasi: Windows 10
	\item Versi Python : Python 3.8.5
\end{enumerate}

\begin{table}[H]			
	\caption{Tabel Pengujian Fungsional}
	\centering
	\begin{tabular}{|p{0.5cm} |p{4cm} |p{5.5cm}| p{3cm}|} \hline
		No. & Aksi Pengguna & Reaksi yang diharapkan & Reaksi Perangkat Lunak\\ \hline     
		1. 	& Mahasiswa menjalankan perangkat lunak & Browser Google Chrome terbuka & Sesuai\\ \hline 
	 		& &  Browser menuju situs Portal Akademik Mahasiswa & Sesuai\\ \hline 
			& &  Browser menuju halaman web untuk perekaman kehadiran daring & Sesuai\\ \hline 
			& &  Melakukan perekaman kehadiran daring secara otomatis & Sesuai\\ \hline
	\end{tabular}
	\label{tab:fungsi}
\end{table}

\subsection{Pengujian Fungsional Dosen}

\textit{Subbab ini ditulis oleh dosen pembimbing.}

Program yang sudah dibuat diujikan juga di komputer dosen pembimbing, untuk melakukan rekam kehadiran pada portal AKUHADIR (subbab \ref{sec:akuhadir}). Berikut adalah spesifikasi komputer yang digunakan untuk menguji:

\begin{enumerate}
	\item \textit{Processor}: Intel Core i3-2120 CPU @ 3.30GHz $\times$ 4
	\item \textit{Random Access Memory} (RAM): 6,0 GiB
	\item Sistem Operasi: Ubuntu 22.04 LTS
	\item Versi Python : Python 3.10.4
\end{enumerate}

Untuk melakukan perekaman kehadiran otomatis di AKUHADIR, perlu penyesuaian \textit{file} \texttt{database.ini} seperti dilihat pada kode \ref{kode:5:databasedosen}.

\begin{lstlisting}[caption=\textit{File} \texttt{database.ini} AKUHADIR (\textit{password} disembunyikan), label=kode:5:databasedosen]
[database_config]
1 = open https://akuhadir.unpar.ac.id
2 = sendkeys #username pascal@unpar.ac.id
3 = click #next_login
4 = sendkeys #password xxx
5 = click button[name=submit]
6 = click a[href='https://akuhadir.unpar.ac.id/absensi?tab=tab2']
7 = click a[onclick='checkin_home()']
8 = quit
\end{lstlisting}

Saat pertama kali dijalankan, program menghasilkan pesan kesalahan seperti ditunjukkan pada kode \ref{kode:5:tkintererror}.

\begin{lstlisting}[caption=Pesan kesalahan program tanpa \textit{tkinter}, label=kode:5:tkintererror]
$ python3 automatedTesting.py 
Traceback (most recent call last):
  File "/home/pascal/Downloads/Dosen/automatedTesting.py", line 11, in <module>
    from tkinter import * 
ModuleNotFoundError: No module named 'tkinter'
\end{lstlisting}

Dengan adanya kesalahan tersebut, dosen juga mengulangi pengujian di komputer lain, yaitu:

\begin{enumerate}
	\item \textit{Processor}: Apple M1
	\item \textit{Random Access Memory} (RAM): 8 GB
	\item Sistem Operasi: macOS Monterey
	\item Versi Python : Python 3.9.10
\end{enumerate}

Saat dijalankan, program menghasilkan pesan kesalahan yang serupa, seperti ditunjukkan pada kode \ref{kode:5:tkintererrormac}.

\begin{lstlisting}[caption=Pesan kesalahan program tanpa \textit{tkinter} pada komputer alternatif, label=kode:5:tkintererrormac]
 % python3 automatedTesting.py
Traceback (most recent call last):
  File "/Users/pascal/Downloads/Dosen/automatedTesting.py", line 13, in <module>
    from tkinter import * 
  File "/opt/homebrew/Cellar/python@3.9/3.9.10/Frameworks/Python.framework/Versions/3.9/lib/python3.9/tkinter/__init__.py", line 37, in <module>
    import _tkinter # If this fails your Python may not be configured for Tk
ModuleNotFoundError: No module named '_tkinter'
\end{lstlisting}

Dari pengamatan pesan kesalahan, didapatkan bahwa program ini membutuhkan pustaka lain, yaitu \texttt{tkinter}. Dalam kasus ini dosen pembimbing tidak melanjutkan \textit{troubleshooting} karena pada paket program yang dikirimkan tidak mengandung instruksi untuk melakukan instalasi pustaka tersebut. Oleh karena itu pengujian fungsional pada komputer dosen dinyatakan tidak berhasil, dan selengkapnya dapat dilihat pada tabel \ref{tab:fungsidosen}.

\begin{table}[H]			
	\caption{Tabel Pengujian Fungsional}
	\centering
	\begin{tabular}{|p{0.5cm} |p{4cm} |p{5.5cm}| p{3cm}|} \hline
		No. & Aksi Pengguna & Reaksi yang diharapkan & Reaksi Perangkat Lunak\\ \hline     
		1. & Dosen menjalankan perangkat lunak & Browser Google Chrome terbuka & Tidak sesuai\\ \hline 
		 	& &  Browser menuju situs AKUHADIR & Tidak sesuai\\ \hline
		 	& &  Browser menuju halaman web untuk perekaman kehadiran & Tidak sesuai\\ \hline
		 	& &  Melakukan perekaman kehadiran daring secara otomatis & Tidak sesuai\\ \hline
	\end{tabular}
	\label{tab:fungsidosen}
\end{table}

\newpage
\subsection{Pengujian Eksperimental}
Pengujian eksperimental dilakukan terhadap beberapa mahasiswa dan dosen Universitas Katolik Parahyangan jurusan Teknik Informatika yang sudah memiliki Google Chrome dan Python3. Metode pengujian dilakukan dengan cara menyebarkan perangkat lunak yang dapat diunduh melalui Google Drive. Setelah menjalankan perangkat lunak tersebut, mahasiswa dan dosen diminta untuk mengisi Google Form untuk mengetahui kelancaran perangkat lunak ketika dijalankan dan mengetahui lama waktu yang dibutuhkan hingga program berhasil melakukan perekaman kehadiran. 

\subsubsection{Hasil Survei Mahasiswa}
Dari 7 responden yang telah mengisi survei, memberi respons bahwa perangkat lunak berjalan dengan baik dan dapat melakukan perekaman kehadiran daring secara otomatis. Hasil diagram lingkaran pada Gambar \ref{fig:noerror} menunjukan bahwa semua responden menyatakan setuju program tidak mengalami \textit{error} atau \textit{crash}.
	\begin{figure}[H]
		\centering
		\includegraphics[scale=0.7]{Gambar/diagramLingkaran.jpg}
		\caption{Diagram Lingkaran Kesetujuan Mahasiswa Terhadap Perangkat Lunak Tidak \textit{Error} atau \textit{Crash}} 
		\label{fig:noerror}
	\end{figure}
Pada Gambar \ref{fig:olMahasiswa} merupakan visualisasi dari hasil survei mengenai lama waktu yang dibutuhkan dari 7 responden(mahasiswa) untuk melakukan perekaman kehadiran daring secara otomatis dengan menjalankan perangkat lunak. Histogram ini dikelompokan berdasarkan rentang waktu per 20 detik. Histogram menunjukan bahwa sebanyak 6 mahasiswa memiliki rentang waktu 0 sampai 20 detik dan sebanyak 1 mahasiswa memiliki rentang waktu 21 sampai 40 detik. 
\begin{figure}[H]
		\centering
		\includegraphics[scale=0.75]{Gambar/HistogramDaringOtomatisMahasiswa.jpg}
		\caption{Histogram Waktu Perekaman Kehadiran Daring Otomatis Mahasiswa}
		\label{fig:olMahasiswa}
	\end{figure}
Hasil survei perekaman kehadiran daring otomatis untuk setiap mahasiswa secara jelas dapat dilihat pada tabel \ref{tab:otomatis} menunjukan waktu yang didapatkan dari 7 responden dalam menjalankan perangkat lunak untuk melakukan perekaman kehadiran daring otomatis. Hasil tabel tersebut menunjukan bahwa dalam melakukan perekaman kehadiran daring otomatis memiliki rata-rata waktu adalah 16,71 detik.
	\begin{table}[ht]			
		\caption{Tabel Perekaman Kehadiran Daring Otomatis Mahasiswa}
		\centering
		\begin{tabular}{|p{3.5cm} |p{7cm}|}
			\hline
			Jumlah Responden &  Waktu Perekaman Kehadiran Otomatis \\ \hline     
			1 orang &  11 detik\\ \hline 
			1 orang &  14 detik\\ \hline 
			1 orang &  15 detik\\ \hline 
			2 orang &  18 detik\\ \hline 
			1 orang &  19 detik\\ \hline 
			1 orang &  22 detik\\ \hline 
		\end{tabular}
		\label{tab:otomatis}
	\end{table}
\\
Hasil survei dari 7 mahasiswa menyatakan setuju bahwa perangkat lunak untuk melakukan perekaman kehadiran daring secara otomatis dapat membuat waktu interaksi dengan situs web atau browser menjadi lebih singkat. Hal tersebut dapat dilihat pada diagram lingkaran pada Gambar \ref{fig:interaksi}.
	\begin{figure}[H]
		\centering
		\includegraphics[scale=0.7]{Gambar/diagramLingkaran.jpg}
		\caption{Diagram Lingkaran Kesetujuan Mahasiswa Terhadap Perangkat Lunak Menghemat Waktu Interaksi dengan Browser} 
		\label{fig:interaksi}
	\end{figure}

	
\subsubsection{Hasil Survei Dosen}
Hasil survei kepada dosen hanya mendapat 1 respon dosen. Hasil diagram lingkaran pada Gambar \ref{fig:noerrorDosen} menunjukan bahwa respon dari seorang dosen menyatakan setuju program tidak mengalami \textit{error} atau \textit{crash}.
\begin{figure}[H]
	\centering
	\includegraphics[scale=0.7]{Gambar/diagramLingkaran.jpg}
	\caption{Diagram Lingkaran Kesetujuan Dosen terhadap Perangkat Lunak Tidak \textit{Error} atau \textit{Crash}} 
	\label{fig:noerrorDosen}
\end{figure}
\newpage
Pada Gambar \ref{fig:olDosen} merupakan visualisasi dari hasil survei mengenai lama waktu yang dibutuhkan dari satu orang dosen untuk melakukan perekaman kehadiran daring secara otomatis dengan menjalankan perangkat lunak. Histogram ini dikelompokan berdasarkan rentang waktu per 20 detik. Histogram menunjukan bahwa sebanyak satu orang dosen memiliki rentang waktu 0 sampai 20 detik. Satu dosen menyatakan waktu yang dibutuhkan untuk melakukan perekaman kehadiran daring secara otomatis adalah 2 detik.
\begin{figure}[H]
	\centering
	\includegraphics[scale=0.75]{Gambar/HistogramDaringOtomatisDosen.jpg}
	\caption{Histogram Waktu Perekaman Kehadiran Daring Otomatis Dosen}
	\label{fig:olDosen}
\end{figure}
Hasil survei dari 1 dosen menyatakan setuju bahwa perangkat lunak untuk melakukan perekaman kehadiran daring secara otomatis dapat membuat waktu interaksi dengan situs web atau browser menjadi lebih singkat. Hal tersebut dapat dilihat pada diagram lingkaran pada Gambar \ref{fig:interaksiDosen}.
\begin{figure}[H]
	\centering
	\includegraphics[scale=0.7]{Gambar/diagramLingkaran.jpg}
	\caption{Diagram Lingkaran Kesetujuan Dosen terhadap Perangkat Lunak Menghemat Waktu Interaksi dengan Browser} 
	\label{fig:interaksiDosen}
\end{figure}

\subsubsection{Perbandingan Hasil Pengujian}
Bagian ini akan membandingkan waktu yang didapatkan dari perekaman kehadiran daring secara otomatis dengan perekaman kehadiran daring dan luring bagi mahasiswa maupun dosen. Tabel \ref{tab:banding} merupakan tabel hasil perbandingan rata-rata waktu untuk melakukan perekaman kehadiran.
\begin{table}[H]			
 	\caption{Tabel Perekaman kehadiran}
 	\centering
 	\begin{tabular}{|p{2cm} |p{4cm} |p{4cm}| p{4cm}|} \hline
 		Pengguna & Rata-rata waktu dengan perangkat lunak & Rata-rata waktu luring & Rata-rata waktu daring\\ \hline     
 		Mahasiswa & 16,71 detik (7 respon)& 7,95 detik (21 respon)& 63 detik (21 respon)\\ \hline 
 		Dosen & 2 detik (1 respon)&  24,33 detik (6 respon)& 31,83 detik (6 respon)\\ \hline 
 	\end{tabular}
 	\label{tab:banding} 
\end{table}
Dari hasil perbandingan pada tabel \ref{tab:banding} secara angka dapat dilihat bahwa untuk mahasiswa waktu perekaman kehadiran dengan perangkat lunak masih lebih lama sedikit dengan waktu perekaman luring tetapi lebih cepat dibanding waktu perekaman daring dan untuk dosen perekaman kehadiran dengan perangkat lunak lebih cepat dibandingkan waktu perekaman daring maupun luring. Perbandingan waktu dari mahasiswa masih kurang akurat karena jumlah respon yang berbeda begitu pula dengan dosen.