\chapter{Kesimpulan dan Saran}
\label{chap:simpulandansaran}

\section{Kesimpulan}
\label{sec:kesimpulan} 
Dari hasil pembangunan perangkat lunak Perekaman Kehadiran Daring Otomatis, didapatkan kesimpulan-kesimpulan sebagai berikut:
\begin{enumerate}
	\item Analisis kebutuhan untuk program perekaman kehadiran daring otomatis dengan melakukan survei terhadap beberapa mahasiswa serta dosen terhadap perekaman kehadiran secara daring.
	\item Model untuk program perekaman kehadiran daring otomatis dengan menggunakan \textit{Command Prommpt} untuk menjalankan program.
	\item Telah berhasil menggunakan \textit{framework} selenium untuk mengimplementasikan fungsi dari program perekaman kehadiran daring otomatis. Program dapat melakukan perekaman kehadiran daring secara otomatis terhadap Portal Akademik Mahasiswa maupun AKUHADIR dengan \textit{framework} selenium.
	\item Telah berhasil membuat program yang mampu secara otomatis melakukan tahap-tahap dalam melakukan perekaman kehadiran secara daring dengan sekali menjalankan perangkat lunak, sehingga waktu interaksi dengan situs webnya menjadi lebih singkat. 
	\item Telah berhasil membangun program perekaman kehadiran daring secara otomatis menggunakan Selenium WebDriver.
\end{enumerate}

\section{Saran}
\label{sec:saran} 
Dari hasil penelitian, pengujian, dan kesimpulan yang didapat, berikut ini adalah beberapa saran untuk pengembang lebih lanjut:
\begin{enumerate}
	\item Melakukan survei lebih banyak lagi untuk perekaman kehadiran daring otomatis agar hasil perbandingannya lebih akurat lagi.
	\item Mempertimbangkan program perekaman kehadiran daring otomatis untuk dosen agar dapat merekam kehadiran mahasiswa, karena terdapat fitur pada AKUHADIR agar dosen dapat melakukan perekaman kehadiran untuk mahasiswa.
\end{enumerate}

