%versi 3 (18-12-2016)
\chapter{\textit{File} Masukan Untuk Perangkat Lunak}
\label{lamp:A}

%terdapat 2 cara untuk memasukkan kode program
% 1. menggunakan perintah \lstinputlisting (kode program ditempatkan di folder yang sama dengan file ini)
% 2. menggunakan environment lstlisting (kode program dituliskan di dalam file ini)
% Perhatikan contoh yang diberikan!!
%
% untuk keduanya, ada parameter yang harus diisi:
% - language: bahasa dari kode program (pilihan: Java, C, C++, PHP, Matlab, C#, HTML, R, Python, SQL, dll)
% - caption: nama file dari kode program yang akan ditampilkan di dokumen akhir
%
% Perhatian: Abaikan warning tentang textasteriskcentered!!
%
\section{\textit{File} Konfigurasi Mahasiswa}
\textit{File} .ini yang digunakan sebagai \textit{file} konfigurasi yang berguna sebagai masukan perangkat lunak perekaman kehadiran daring secara otomatis bagi mahasiswa.
\lstinputlisting[language=, caption=database.ini (\textit{password disembunyikan})]{./Lampiran/database.ini} 

\section{\textit{File} Konfigurasi Dosen}
\textit{File} .ini yang digunakan sebagai \textit{file} konfigurasi yang berguna sebagai masukan perangkat lunak perekaman kehadiran daring secara otomatis bagi dosen.
\lstinputlisting[language=, caption=database.ini (\textit{password disembunyikan})]{./Lampiran/databaseDosen.ini} 

