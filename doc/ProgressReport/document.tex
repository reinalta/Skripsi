\documentclass[a4paper,twoside]{article}
\usepackage[T1]{fontenc}
\usepackage[bahasa]{babel}
\usepackage{graphicx}
\usepackage{graphics}
\usepackage{float}
\usepackage[cm]{fullpage}
\pagestyle{myheadings}
\usepackage{etoolbox}
\usepackage{setspace} 
\usepackage{lipsum} 
\setlength{\headsep}{30pt}
\usepackage[inner=2cm,outer=2.5cm,top=2.5cm,bottom=2cm]{geometry} %margin
% \pagestyle{empty}

\makeatletter
\renewcommand{\@maketitle} {\begin{center} {\LARGE \textbf{ \textsc{\@title}} \par} \bigskip {\large \textbf{\textsc{\@author}} }\end{center} }
\renewcommand{\thispagestyle}[1]{}
\markright{\textbf{\textsc{Laporan Perkembangan Pengerjaan Skripsi\textemdash Sem. Genap 2015/2016}}}

\onehalfspacing
 
\begin{document}

\title{\@judultopik}
\author{\nama \textendash \@npm} 

%ISILAH DATA BERIKUT INI:
\newcommand{\nama}{Reinalta Sugianto}
\newcommand{\@npm}{2017730035}
\newcommand{\tanggal}{13/12/2021} %Tanggal pembuatan dokumen
\newcommand{\@judultopik}{Perekaman Kehadiran Daring Otomatis} % Judul/topik anda
\newcommand{\kodetopik}{PAN5191}
\newcommand{\jumpemb}{1} % Jumlah pembimbing, 1 atau 2
\newcommand{\pembA}{Pascal Alfadian Nugroho}
\newcommand{\pembB}{-}
\newcommand{\semesterPertama}{51 - Ganjil 21/22} % semester pertama kali topik diambil, angka 1 dimulai dari sem Ganjil 96/97
\newcommand{\lamaSkripsi}{1} % Jumlah semester untuk mengerjakan skripsi s.d. dokumen ini dibuat
\newcommand{\kulPertama}{Skripsi 1} % Kuliah dimana topik ini diambil pertama kali
\newcommand{\tipePR}{B} % tipe progress report :
% A : dokumen pendukung untuk pengambilan ke-2 di Skripsi 1
% B : dokumen untuk reviewer pada presentasi dan review Skripsi 1
% C : dokumen pendukung untuk pengambilan ke-2 di Skripsi 2

% Dokumen hasil template ini harus dicetak bolak-balik !!!!

\maketitle

\pagenumbering{arabic}

\section{Data Skripsi} %TIDAK PERLU MENGUBAH BAGIAN INI !!!
Pembimbing utama/tunggal: {\bf \pembA}\\
Pembimbing pendamping: {\bf \pembB}\\
Kode Topik : {\bf \kodetopik}\\
Topik ini sudah dikerjakan selama : {\bf \lamaSkripsi} semester\\
Pengambilan pertama kali topik ini pada : Semester {\bf \semesterPertama} \\
Pengambilan pertama kali topik ini di kuliah : {\bf \kulPertama} \\
Tipe Laporan : {\bf \tipePR} -
\ifdefstring{\tipePR}{A}{
			Dokumen pendukung untuk {\BF pengambilan ke-2 di Skripsi 1} }
		{
		\ifdefstring{\tipePR}{B} {
				Dokumen untuk reviewer pada presentasi dan {\bf review Skripsi 1}}
			{	Dokumen pendukung untuk {\bf pengambilan ke-2 di Skripsi 2}}
		}
		
\section{Latar Belakang}
Perkuliahan di UNPAR biasanya membutuhkan perekaman kehadiran untuk mengetahui kehadiran mahasiswa dan dosen, bagi mahasiswa UNPAR perekaman kehadiran biasanya dilakukan dengan melakukan tanda tangan pada daftar kehadiran atau dicatat langsung oleh dosen yang memanggil mahasiswanya, sedangkan bagi dosen UNPAR perekaman kehadiran dilakukan dengan menggunakan  \textit{fingerprint}. Perekaman kehadiran diperkirakan membutuhkan waktu sekitar kurang dari 5 detik.

Pada tahun 2020 terjadi pandemi Covid-19 di seluruh negara. Pandemi Covid-19 masuk ke Indonesia pada awal bulan Maret tahun 2020. Covid-19 adalah penyakit yang disebabkan oleh virus \textit{severe acute respiratory syndrome coronavirus 2} (SARS-CoV-2). Penularan virus Covid-19 terjadi saat seseorang menyentuh barang yang sudah terkontaminasi oleh droplet orang yang terkena virus Covid-19 atau terkena droplet orang lain saat berinteraksi langsung dengan orang yang terkena virus Covid-19.  Akibat pandemi Covid-19 yang dapat menular ini, maka hampir seluruh kegiatan di Indonesia dilakukan secara daring untuk mengurangi interaksi orang secara langsung yang dapat meningkatkan angka penularan virus tersebut. 

Pembelajaran secara daring diberlakukan oleh UNPAR di akhir bulan Maret untuk seluruh kegiatan perkuliahan demi mencegah penularan virus Covid-19. Akibat diberlakukannya pembelajaran secara daring, maka perekaman kehadiran di UNPAR dilakukan dengan menggunakan aplikasi atau situs web milik UNPAR. Cara perekaman kehadiran secara daring di UNPAR ini mumbutuhkan waktu lebih agar dapat tercatat perekaman kehadirannya, karena butuh waktu untuk membuka situs web serta perlu memasukan \textit{email} dan \textit{password} hingga akhirnya melakukan perekaman kehadiran. 

Selenium adalah \textit{open-source} \textit{framework} pengujian otomatisasi untuk aplikasi web\cite{selenium}. WebDriver menggunakan API otomatisasi \textit{browser} yang disediakan oleh vendor \textit{browser} untuk mengontrol \textit{browser} dan melakukan pengujian. API WebDriver ini seolah-olah membuat pengguna secara langsung mengoperasi \textit{browser}, padahal dijalankan secara otomatis langsung oleh \textit{API} WebDriver tersebut. Selenium WebDriver adalah sebuah \textit{tools} yang berguna untuk melakukan otomatisasi terhadap web pada \textit{browser}. Selenium WebDriver ini tersedia untuk bahasa pemrograman Ruby, Java, Python, C\#, dan JavaScript. Pembuatan Perekaman kehadiran daring otomatis ini akan menggunakan Selenium WebDriver dengan bahasa pemrograman Python.  

Pada skripsi ini, akan dibuat sebuah perangkat lunak yang dapat melakukan perekaman kehadiran otomatis dengan sistem menerima rangsangan satu ``klik'' sehingga dapat melakukan hal-hal berikut :
\begin{enumerate}
	\item Membuka peramban.
	\item Membuka situs web perekaman kehadiran.
	\item Mengisi dan \textit{login} dengan \textit{username} serta \textit{password} yang ddiambil dari file konfigurasi.
	\item Melakukan rekam kehadiran.
\end{enumerate} 
perangkat lunak ini bertujuan agar mahasiswa dan dosen dapat melakukan perekaman kehadiran secara online dengan lebih mudah serta mengurangi waktu yang dibutuhkan untuk berinteraksi dengan aplikasi atau situs web dan bukan untuk mempercepat waktu agar kehadiran terekam, sehingga membuat waktu perekaman kehadiran secara daring dapat menyamai waktu perekaman kehadiran secara luring. 

\section{Rumusan Masalah}
Rumusan masalah yang akan dibahas di skripsi ini adalah sebagai berikut :
\begin{itemize}
	\item Bagaimana cara membangun program Perekaman Kehadiran Daring Otomatis?
	\item Bagaimana cara mengurangi waktu interaksi dengan aplikasi atau situs web untuk merekam kehadiran?
\end{itemize}

\section{Tujuan}
Tujuan yang ingin dicapai dari penulisan skripsi ini sebagai berikut :
\begin{itemize}
	\item Membangun program menggunakan Selenium WebDriver.
	\item Membuat program yang mampu menerima rangsangan satu tombol untuk melakukan beberapa hal menggunakan Selenium.
\end{itemize}

\section{Detail Perkembangan Pengerjaan Skripsi}
Detail bagian pekerjaan skripsi sesuai dengan rencan kerja/laporan perkembangan terkahir :
	\begin{enumerate}
		\item \textbf{Melakukan studi mengenai Selenium WebDriver.}\\
		{\bf Status :} Ada sejak rencana kerja skripsi.\\
		{\bf Hasil :} 
		
		\item \textbf{Mempelajari bahasa pemrograman python.}\\
		{\bf Status :} Ada sejak rencana kerja skripsi.\\
		{\bf Hasil :}

		\item \textbf{Mempelajari cara menggunakan Selenium.}\\
		{\bf Status :} Ada sejak rencana kerja skripsi.\\
		{\bf Hasil :}

		\item \textbf{Menganalisis web Student Portal UNPAR.}\\
		{\bf Status :} Ada sejak rencana kerja skripsi.\\
		{\bf Hasil :}

		\item \textbf{Membangun program perekaman kehadiran daring otomatis.}\\
		{\bf Status :} Ada sejak rencana kerja skripsi.\\
		{\bf Hasil :}

		\item \textbf{Melakukan pengujian dan eksperimen.}\\
		{\bf Status :} Ada sejak rencana kerja skripsi. \\
		{\bf Hasil :} 

		\item \textbf{Menulis dokumen skripsi.} \\
		{\bf Status :} Ada sejak rencana kerja skripsi.\\
		{\bf Hasil :}
		

	\end{enumerate}

\section{Pencapaian Rencana Kerja}
Langkah-langkah kerja yang berhasil diselesaikan dalam Skripsi 1 ini adalah sebagai berikut:
\begin{enumerate}
\item Melakukan studi mengenai Selenium WebDriver. 
\item Mempelajari bahasa pemrograman python.
\item Mempelajari cara menggunakan Selenium.
\item Menganalisis web Student Portal UNPAR.
\item Menulis dokumen skripsi Bab 1-3.
\end{enumerate}



\vspace{1cm}
\centering Bandung, \tanggal\\
\vspace{2cm} \nama \\ 
\vspace{1cm}

Menyetujui, \\
\ifdefstring{\jumpemb}{2}{
\vspace{1.5cm}
\begin{centering} Menyetujui,\\ \end{centering} \vspace{0.75cm}
\begin{minipage}[b]{0.45\linewidth}
% \centering Bandung, \makebox[0.5cm]{\hrulefill}/\makebox[0.5cm]{\hrulefill}/2013 \\
\vspace{2cm} Nama: \pembA \\ Pembimbing Utama
\end{minipage} \hspace{0.5cm}
\begin{minipage}[b]{0.45\linewidth}
% \centering Bandung, \makebox[0.5cm]{\hrulefill}/\makebox[0.5cm]{\hrulefill}/2013\\
\vspace{2cm} Nama: \pembB \\ Pembimbing Pendamping
\end{minipage}
\vspace{0.5cm}
}{
% \centering Bandung, \makebox[0.5cm]{\hrulefill}/\makebox[0.5cm]{\hrulefill}/2013\\
\vspace{2cm} Nama: \pembA \\ Pembimbing Tunggal
}
\end{document}

